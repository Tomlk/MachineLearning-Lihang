\documentclass{article}
\usepackage[T1]{fontenc}
\usepackage[UTF8]{ctex} %中文支持
\usepackage[utf8]{inputenc}
\usepackage{microtype}
\usepackage{lmodern}
\usepackage{amsmath}
\usepackage{graphics}
\usepackage{algorithm}
\usepackage{algorithmicx}
\usepackage{ulem}
\usepackage{color}
\usepackage[backend=bibtex,
                style=authoryear,
                natbib=true, 
                style=numeric-comp
                ]{biblatex}    
\usepackage[english]{babel}
\usepackage{filecontents}
\begin{filecontents*}{\jobname.bib}
  @book{ASH,
    author = {Seiffert, U. and Wech, L.},
    year = {2003},
    title = {Automotive Safety Handbook},
  }
\end{filecontents*}
\addbibresource{\jobname.bib}
\begin{document}
\renewcommand\maketitle{}
\section{线性可分支持向量机(硬间隔最大化)}
对于约束最优化问题:
\begin{gather}   \label{eq:m}
    \min\limits_{w,b} \frac{1}{2}||w||^2\\
    s.t\quad y_i(w\cdot x_i+b)-1 \ge 0,\quad i=1,2,\dots,N  \label{eq:n}
\end{gather}
易知式(\ref{eq:n})满足等号“=”的样本点实例为{\color{red}支持向量}。

运用拉格朗日乘数法,定义拉格朗日函数:
\begin{align}
    L(w,b,\alpha)&=\frac{1}{2}||w||^2-\sum_{i=1}^{N}\alpha_i\cdot(y_i (w\cdot x_i+b)-1)\\
    &=\frac{1}{2}||w||^2-\sum_{i=1}^{N}\alpha_i y_i (w\cdot x_i+b)+\sum_{i=1}^{N}\alpha_i  \label{eq:1}
\end{align}

即该问题是来探究$\max\limits_{\alpha}\min\limits_{w,b}L(w,b,\alpha)$,因此需要先求$L(w,b,\alpha)$对$w,b$的极小,再求对$\alpha$的极大.

(1)先求$L(w,b,\alpha)$对$w,b$的极小
 根据式(\ref{eq:1})分别对w,b求偏导数并令其等于0

对w,
\begin{gather}
    \bigtriangledown_w L(w,b,\alpha)=w-\sum_{i=1}^{N}\alpha_i y_i x_i=0    
\end{gather}
对b,
\begin{gather}
    \bigtriangledown_b L(w,b,\alpha)=-\sum_{i=1}^{N}\alpha_i y_i=0    
\end{gather}

因此有
\begin{gather}
    w=\sum_{i=1}^{N}\alpha_i y_i x_i \label{eq:2} \\
    \sum_{i=1}^{N}\alpha_i y_i=0    \label{eq:3}
\end{gather}

 把式(\ref{eq:2})代入(\ref{eq:1})可得到如下公式:
\begin{equation}
    L(w,b,\alpha)=\frac{1}{2}\sum_{i=1}^{N} \sum_{j=1}^{N} \alpha_i \alpha_j y_i y_j (x_i\cdot x_j)-\sum_{i=1}^{N} \alpha_i y_i((\sum_{j=1}^{N}\alpha_j y_j x_j)\cdot x_i +b)+\sum_{i=1}^{N}\alpha_i  \label{eq:4}
\end{equation}
 又因为式(\ref{eq:3}),因此式(\ref{eq:4})的第二式中$\sum_{i=1}^{N}\alpha_i y_ib=0$,因此有:
\begin{align}
    L(w,b,\alpha)&=\frac{1}{2}\sum_{i=1}^{N} \sum_{j=1}^{N} \alpha_i \alpha_j y_i y_j (x_i\cdot x_j)-\sum_{i=1}^{N} \alpha_i y_i((\sum_{j=1}^{N}\alpha_j y_j x_j)\cdot x_i )+\sum_{i=1}^{N}\alpha_i\\
    &=\frac{1}{2}\sum_{i=1}^{N} \sum_{j=1}^{N} \alpha_i \alpha_j y_i y_j (x_i\cdot x_j)-\sum_{i=1}^{N} \sum_{j=1}^{N} \alpha_i \alpha_j y_i y_j (x_i\cdot x_j)+\sum_{i=1}^{N}\alpha_i\\
    &=-\frac{1}{2}\sum_{i=1}^{N} \sum_{j=1}^{N} \alpha_i \alpha_j y_i y_j (x_i\cdot x_j)+\sum_{i=1}^{N}\alpha_i
\end{align}
即
\begin{equation}
    \min\limits_{w,b}L(w,b,\alpha)=-\frac{1}{2}\sum_{i=1}^{N} \sum_{j=1}^{N} \alpha_i \alpha_j y_i y_j (x_i\cdot x_j)+\sum_{i=1}^{N}\alpha_i
\end{equation}

(2)再求$\min\limits_{w,b}L(w,b,\alpha)$对$\alpha$的极大,即:
\begin{gather}
    \max\limits_{\alpha} \quad -\frac{1}{2}\sum_{i=1}^{N} \sum_{j=1}^{N} \alpha_i \alpha_j y_i y_j (x_i\cdot x_j)+\sum_{i=1}^{N}\alpha_i  \label{eq:5}\\
    s.t \quad \sum_{i=1}^{N} \alpha_i y_i=0\\
    \alpha_i \ge 0,\quad i=1,2,\dots,N
\end{gather}

将式(\ref{eq:5})的目标函数由求极大值转换为求极小值,就得到下面与之等价的对偶问题:
\begin{gather}
    \min\limits_{\alpha} \quad \frac{1}{2}\sum_{i=1}^{N} \sum_{j=1}^{N} \alpha_i \alpha_j y_i y_j (x_i\cdot x_j)-\sum_{i=1}^{N}\alpha_i  \\
    s.t \quad \sum_{i=1}^{N} \alpha_i y_i=0\\
    \alpha_i \ge 0,\quad i=1,2,\dots,N
\end{gather}
我们需要{\color{red}求解}(利用{\color{red}SMO}算法))得到最优解{\color{red} $\alpha^*=(\alpha_1^*,\alpha_2^*,\dots,\alpha_N^*)^T$}
我们选择一个$\alpha_j^*>0$,
这样我们能够求得原始目标优化问题(\ref{eq:m})的解为:

\begin{gather}
    w^*=\sum_{i=1}^{N}\alpha_i^* y_i x_i\\
    b^*=y_j-\sum_{i=1}^{N}\alpha_i^* y_i(x_i\cdot x_j)
\end{gather}

于是可以求得分离超平面及分类决策函数
\begin{gather}
    w^*\cdot x+b^*=0\\
    f(x)=sign(w^*\cdot x+b^*)
\end{gather}

\section{线性支持向量机(软间隔最大化)}
其中凸二次优化的原始问题为:
\begin{gather}  \label{eq:z}
    \min\limits_{w,b,\xi} \quad \frac{1}{2}||w||^2+C\sum_{i=1}^{N}\xi_i\\
    s.t.\quad y_i(w\cdot x_i+b)\ge1-\xi_i,\quad i=1,2,\dots,N\\
    \xi_i\ge0,\quad i=1,2,\dots,N
\end{gather}

我们需要求解该原始问题(\ref{eq:z})的对偶问题,先写出拉格朗日函数:
\begin{equation}  \label{eq:ll}
    L(w,b,\xi,\alpha,\mu)=\frac{1}{2}||w||^2+C\sum_{i=1}^{N}\xi_i-\sum_{i=1}^{N}\alpha_i(y_i(w\cdot x_i)-1+\xi_i)-\sum_{i=1}^{N}\mu_i\xi_i
\end{equation}
其中,$\alpha_i\ge0,\mu_i\ge0$.

与前面同理,先求$L(w,b,\xi,\alpha,\mu)$对$w,b,\xi$的极小:
\begin{gather}
    \bigtriangledown_w L(w,b,\xi,\alpha,\mu)=w-\sum_{i=1}^{N}\alpha_i y_i x_i=0\\
    \bigtriangledown_b L(w,b,\xi,\alpha,\mu)=-\sum_{i=1}^{N}\alpha_i y_i=0 \\
    \bigtriangledown_{\xi_i} L(w,b,\xi,\alpha,\mu)=C-\alpha_i -\mu_i=0
\end{gather}
这样,可以得到:
\begin{gather}
    w=\sum_{i=1}^{N}\alpha_i y_i x_i \label{eq:d1}\\ 
    \sum_{i=1}^{N}\alpha_i y_i=0 \label{eq:d2}\\
    C-\alpha_i-\mu_i=0 \label{eq:d3}
\end{gather}

将式(\ref{eq:d1}),(\ref{eq:d2}),(\ref{eq:d3})代入式(\ref{eq:ll}),可得到
\begin{equation}
    \min\limits_{w,b,\xi} L(w,b,\xi,\alpha,\mu)=-\frac{1}{2}\sum_{i=1}^{N}\sum_{j=1}^{N}\alpha_i \alpha_j y_i y_j(x_i\cdot x_j)+\sum_{i=1}^{N}
\end{equation}
再全对$\min\limits_{w,b,\xi} L(w,b,\xi,\alpha,\mu)$ 求$\alpha$ 的极大,即得到对偶问题:
\begin{gather}
    \max\limits_{\alpha}\quad -\frac{1}{2}\sum_{i=1}^{N}\sum_{j=1}^{N}\alpha_i \alpha_j y_i y_j(x_i\cdot x_j)+\sum_{i=1}^{N}\alpha_i\\
    s.t.\quad \sum_{i=1}^{N}\alpha_i y_i=0\\
    C-\alpha_i-\mu_i=0   \label{eq:mu}\\
    \alpha_i\ge0\\
    \mu_i\ge0,\quad i=1,2,\dots,N
\end{gather}
利用式(\ref{eq:mu})消去$\mu_i$,可得到如下对偶问题:
\begin{gather}    
    \min\limits_{\alpha}\quad \frac{1}{2}\sum_{i=1}^{N}\sum_{j=1}^{N}\alpha_i \alpha_j y_i y_j(x_i\cdot x_j)-\sum_{i=1}^{N}\alpha_i\\
    s.t.\quad \sum_{i=1}^{N}\alpha_i y_i=0\\
    0\le \alpha_i \le C,\quad i=1,2,\dots,N
\end{gather}
\uline{这里,C>0成为惩罚参数,一般由应用问题决定,C值大时对误分类的惩罚增大,C值小时对误分类的惩罚减小} 
% \uline{这里,C>0成为惩罚参数,一般由应用问题决定,C值大时对误分类的惩罚增大,C值小时对误分类的惩罚减小}。

目标是:{\color{red}使间隔尽量大同时使误分类点的个数尽量小}

分类超平面及其决策函数与线性可分线性向量机的相似,只是分量选择为$0\le \alpha_j^* \le C$。
\section{序列最小最优化算法(Sequential minimal optimization)SMO 算法来求解$\alpha$}
SMO算法是一种启发式算法,如果所有变量的解都满足此最优问题的KKT条件(Karush-Kuhn-Tucker\ conditions),那么这个最优化问题的解就得到了。

在这里用非线性支持向量机学习算法:即SMO算法要求解如下凸二次优化问题:
\begin{gather}    \label{eq:tuquestion}
    \min\limits_{\alpha}\quad \frac{1}{2}\sum_{i=1}^{N}\sum_{j=1}^{N}\alpha_i \alpha_j y_i y_j K(x_i,x_j)-\sum_{i=1}^{N}\alpha_i\\
    s.t.\quad \sum_{i=1}^{N}\alpha_i y_i=0\\
    0\le \alpha_i \le C,\quad i=1,2,\dots,N
\end{gather}


整个SMO算法包括两部分:

1.{\color{red}求解两个变量的二次规划的解析方法}

2.{\color{red}选择变量的启发式方法}

\subsection{两个变量二次规划的求解方法}
不失一般性,将假设选择的$\alpha_1$与$\alpha_2$作为自变量,而将其它$\alpha_i(i=3,4,...,N)$进行固定,这样的(\ref{eq:tuquestion})优化问题可以转写为
\begin{gather}
    \min\limits_{\alpha_1,\alpha_2}\quad W(\alpha_1,\alpha_2)=\frac{1}{2}\alpha_1^2 y_1^2 K_{11}+\frac{1}{2}\alpha_2^2 y_2^2 K_{22}+\frac{1}{2}\alpha_1 \alpha_2 y_1 y_2 K_{12}+\frac{1}{2} \alpha_2 \alpha_1 y_2 y_1 K_{21}-\sum_{i=1}^{N}\alpha_i \\+\frac{1}{2}\alpha_1 y_1 \sum_{i=3}^{N} y_i \alpha_i (K_{i1}+K_{1i})+\frac{1}{2} \alpha_2 y_2 \sum_{i=3}^{N} y_i \alpha_i (K_{i2}+K_{2i})+\frac{1}{2}\sum_{i=3}^{N}\sum_{j=3}^{N}\alpha_i \alpha_j y_i y_j K_{ij} 
\end{gather}
问题可以转换为
\begin{gather}
    \min\limits_{\alpha_1,\alpha_2}\quad W(\alpha_1,\alpha_2)=\frac{1}{2}K_{11}\alpha_1^2+\frac{1}{2}K_{22}\alpha_2^2+y_1 y_2 K_{12} \alpha_1 \alpha_2 \nonumber\\
    -(\alpha_1+\alpha_2) +y_1 \alpha_1\sum_{i=3}^{N} y_i \alpha_i K_{i1} +y_2 \alpha_2 \sum_{i=3}^{N}y_i \alpha_i K_{i2}  \label{eq:temption}\\
    s.t.\quad \alpha_i y_i+\alpha_2 y_2=-\sum_{i=3}^{N}y_i \alpha_i=\varsigma   \label{eq:relation12}\\
    0\le \alpha_i \le C,\quad i=1,2
\end{gather}
式(\ref{eq:relation12})同时乘上$y_1$
\begin{gather}
    \alpha_1 y_1^2+\alpha_2 y_1 y_2=y_1 \varsigma\\
    \alpha_1=y_1(\varsigma-\alpha_2 y_2)
\end{gather}

代入(\ref{eq:temption}),
\begin{gather}
    W(\alpha_2)=\frac{1}{2} K_{11}(\varsigma-\alpha_2 y_2)^2 +\frac{1}{2}K_{22}\alpha_2^2+K_{12} y_2 (\varsigma-\alpha_2 y_2)\alpha_2-y_1 \varsigma+(y_1 y_2-1) \alpha_2 \nonumber\\
    +(\varsigma-\alpha_2 y_2) \sum_{i=1}^{N}y_i \alpha_i K_{i1}+y_2 \alpha_2 \sum_{i=3}^{N}y_i \alpha_i K_{i2}
\end{gather}
在这里,令$v_1=\sum_{i=1}^{N}y_i \alpha_i K_{i1}$,$v_2=\sum_{i=3}^{N}y_i \alpha_i K_{i2}$
上式可化简为
\begin{gather}
    W(\alpha_2)=\frac{1}{2} K_{11}(\varsigma-\alpha_2 y_2)^2 +\frac{1}{2}K_{22}\alpha_2^2+K_{12} y_2 (\varsigma-\alpha_2 y_2)\alpha_2 \nonumber\\
    -y_1 (\varsigma-\alpha_2 y_2)- \alpha_2+v_1(\varsigma-\alpha_2 y_2)+y_2 v_2 \alpha_2 
\end{gather}

$W(\alpha_2)$对$\alpha_2$求导数得到:
\begin{gather}
    \frac{\partial W}{\partial \alpha_2}=K_{11}y_2(\alpha_2 y_2 -\varepsilon)+K_{22}\alpha_2 +y_2 K_{12}\varsigma-2K_{12}\alpha_2+y_1 y_2 -1-v_1y_2+y_2 v_2 \nonumber\\
    =K_{11}\alpha_2 +K_{22}\alpha_2 -2K_{12}\alpha_2-K_{11}\varsigma y_2+K_{12}\varsigma y_2+y_1 y_2 -1-v_1y_2+y_2 v_2 \label{eq:partial}
\end{gather}
令其为0,又考虑到SVM对数据点的预测值为
\begin{equation}
    {\color{red}g(x)=\sum_{i=1}^{N}\alpha_i y_i K(x_i,x) +b}
\end{equation}
在此,令
\begin{equation}
    {\color{red}E_i=g(x_i)-y_i=(\sum_{j=1}^{N}\alpha_j y_j K(x_j,x_i) +b)-y_i,\quad i=1,2} \label{eq:E}
\end{equation}
令(\ref{eq:partial})为0,并将$\varsigma=\alpha_1^{old} y_1+\alpha_2^{old} y_2$代入,
得到
\begin{equation}
    (K_{11}+K_{22}-2K_{12})\alpha_2^{new,unc}=(K_{11}+K_{22}-2K_{12})\alpha_2^{old}+y_2(E_1-E_2)
\end{equation}
将$\eta=K_{11}+K_{22}-2K_{12}$ 代入,有
\begin{equation}
    \alpha_2^{new,unc}=\alpha_2^{old}+\frac{y_2(E_1-E_2)}{\eta}
\end{equation}


还需要对原始解进行修剪

当$y_1\neq y_2$时,上下界表示为:

下界:$L=max(0,\alpha_2^{old}-\alpha_1^{pld})$\quad
上界:$H=min(C,C+\alpha_2^{old}-\alpha_1^{old})$

当$y_1\neq y_2$时,上下界表示为:

下界:$L=max(0,\alpha_2^{old}+\alpha_1^{pld}-C)$\quad
上界:$H=min(C,\alpha_2^{old}+\alpha_1^{old})$

经过修剪后的$\alpha_2$:
\[
  \alpha_2^{new} = 
  \begin{cases}
    H, &\text{$\alpha_2^{new,unclipped}>H$}\\
    \alpha_2^{new,unclipped},&\text{$L\le\alpha_2^{new,unclipped}\le H$}\\
	L, &\text{$\alpha_2^{new,unclipped}<L$}
  \end{cases}
\]

又因为公式
\begin{equation}
    \alpha_1^{old}y_1 +\alpha_2^{old}y_2=\alpha_1^{new}y_1 +\alpha_2^{new}y_2=\varsigma
\end{equation}
我们可以计算出$\alpha_1^{new}$

\subsection{变量选择的方法}
\subsubsection{阈值b的更新}
当$0<\alpha_1^{new}<C$时,
\begin{equation}
    b_1^{new}=y_1-\sum_{i=3}^{N}\alpha_i y_i K_{i1}-\alpha_1^{new} y_1 K_{11}-\alpha_2^{new} y_2 K_{21}\label{eq:b1new}
\end{equation}
由式(\ref{eq:E})式我们可以得到
\begin{align}
    E_1&=(\sum_{j=1}^{N}\alpha_j y_j K(x_j,x_1)+b^{old})-y_1\\
    &=\sum_{i=3}^{N}\alpha_i y_i K_{i1}+\alpha_1^{old} y_1 K_{11}+\alpha_2^{old} y_2 K_{21}+b^{old}-y_1
\end{align}
由此可得
\begin{equation}
    y_1-\sum_{i=3}^{N}\alpha_i y_i K_{i1}=-E_1+\alpha_1^{old} y_1 K_{11} +\alpha_2^{old} y_2 K_{21} +b^{old}
\end{equation}
代入式(\ref{eq:b1new}),得
\begin{equation}
    b_1^{new}=-E_1-y_1 K_{11}(\alpha_1^{new}-\alpha_1^{old})-y_2 K_{21}(\alpha_2^{new}-\alpha_2^{old})+b^{old}
\end{equation}
同理,当$0<\alpha_2^{new}<C$时,有:
\begin{equation}
    b_2^{new}=-E_2-y_1 K_{12}(\alpha_1^{new}-\alpha_1^{old})-y_2 K_{22}(\alpha_2^{new}-\alpha_2^{old})+b^{old}
\end{equation}

并且,每次对两个变量进行优化之后,还需要更新对应的$E_i$值,并将它们保存在列表中:
\begin{equation}
    E_i^{new}=\sum_{S} y_j \alpha_j K(x_i,x_j)+b^{new}-y_i
\end{equation}
其中,S是所有支持向量的集合
\end{document}


